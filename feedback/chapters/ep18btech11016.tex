\begin{enumerate}[label=\thesection.\arabic*.,ref=\thesection.\theenumi]
\numberwithin{equation}{enumi}

\item The feedback current amplifier in Fig. \ref{fig:ep18btech11016_circuit} utilizes an op amp with an input differential resistance $R_{id}$, an open-loop gain $\mu$ , and an output resistance $r_o$. The output current $I_o$ that is delivered to the load resistance $R_L$ is sensed by the feedback network composed of the two resistances $R_M$ and $R_F$ and a fraction $I_f$, is fed back to the amplifier input node. Find expressions for $G = \cfrac{I_o}{I_i}$, $H = \cfrac{I_f}{I_o}$ and $T = \cfrac{I_o}{I_s}$, assuming that the feedback causes the voltage at the input node to be near ground. If the loop gain is large, what does the closed-loop current gain become? State precisely the condition under which this is obtained. For $\mu$ = $10^4$ V/V, $R_{id}$ = 1 M$\Omega$, $r_o$ = 100 $\Omega$, $R_L$ = 10 k$\Omega$ , $R_M$ = 100 $\Omega$ , and $R_F$ = 10 k$\Omega$, find G, H, and T.
\\
\solution Follow the below sub-questions for solution.
\\

%--------------------------------------------------------%

\item Fig. \ref{fig:ep18btech11016_circuit} shows a feedback current amplifier. Draw the equivalent control system.
\\
\solution See fig \ref{fig:ep18btech11016_block}
\\

\renewcommand{\thefigure}{\theenumi.\arabic{figure}}
\begin{figure}[!ht]
	\begin{center}
		\resizebox{\columnwidth}{!}{\begin{circuitikz}
\ctikzset{bipoles/length=1cm}

\draw 
(0, 0) node[op amp] (opamp) {$\mu$}
(opamp.+) to (-1,-0.35)--(-1,-0.7) node[ground]{}
(opamp.-) --(-1.5,0.35) to [short, i<=\small$I_i$] (-1.5,1.5) to[R=\small$R_F$, i>_=\small$I_f$, *-*] (1.5,1.5) to[R=\small$R_L$, i<=\small$I_o$] (1.5,0)
(opamp.out) --(1.5, 0)
(1.5, 1.5) to[R=\small$R_M$] (1.5, 2.9) to (2, 2.9) node[ground]{}
(-1.5, 2.9) to[I_=\small$I_s$] (-1.5, 1.5) 
(-1.5, 2.9) --(-1, 2.9) node[ground]{}

;\end{circuitikz}}
	\end{center}
\caption{}
\label{fig:ep18btech11016_circuit}
\end{figure}
%
\begin{figure}[!ht]
	\begin{center}
		\resizebox{\columnwidth}{!}{\tikzstyle{input} = [coordinate]
\tikzstyle{output} = [coordinate]
\tikzstyle{block} = [draw, rectangle]
\tikzstyle{sum} = [draw, circle]

\begin{tikzpicture}[auto, node distance=2cm,>=latex']
    \node [input, name=input] {$I_s$};
    \node [sum, right of=input] (sum) {};
    \node [block, right of=sum] (controller) {$G$};
    \node [output, right of=controller] (output) {};
    \node [block, below of=controller] (feedback) {$H$};
    \draw [draw,->] (input) -- node {$I_{s}$} (sum);
    \draw [->] (sum) -- node {$I_{i}$} (controller);
    \draw [->] (controller) -- node [name=y] {$I_o$}(output);
    \draw [->] (y) |- (feedback);
    \draw [->] (feedback) -| node[pos=0.95]{$-$}  node [near end] {$I_f$} (sum);
    \draw [->] (feedback) -| node[pos=1.15]{$+$}  node [near end] {} (sum);
\end{tikzpicture}}
	\end{center}
\caption{}
\label{fig:ep18btech11016_block}
\end{figure}
\renewcommand{\thefigure}{\theenumi}

%--------------------------------------------------------%

\item Refer table \ref{table:ep18btech11016_comps} for the parameters and draw the small signal equivalent model of the fig \ref{fig:ep18btech11016_circuit}
\\
\solution See fig \ref{fig:ep18btech11016_equi}
\\

\begin{table}[!ht]
    \centering
    %%%%%%%%%%%%%%%%%%%%%%%%%%%%%%%%%%%%%%%%%%%%%%%%%%%%%%%%%%%%%%%%%%%%%%
%%                                                                  %%
%%  This is the header of a LaTeX2e file exported from Gnumeric.    %%
%%                                                                  %%
%%  This file can be compiled as it stands or included in another   %%
%%  LaTeX document. The table is based on the longtable package so  %%
%%  the longtable options (headers, footers...) can be set in the   %%
%%  preamble section below (see PRAMBLE).                           %%
%%                                                                  %%
%%  To include the file in another, the following two lines must be %%
%%  in the including file:                                          %%
%%        \def\inputGnumericTable{}                                 %%
%%  at the beginning of the file and:                               %%
%%        \input{name-of-this-file.tex}                             %%
%%  where the table is to be placed. Note also that the including   %%
%%  file must use the following packages for the table to be        %%
%%  rendered correctly:                                             %%
%%    \usepackage[latin1]{inputenc}                                 %%
%%    \usepackage{color}                                            %%
%%    \usepackage{array}                                            %%
%%    \usepackage{longtable}                                        %%
%%    \usepackage{calc}                                             %%
%%    \usepackage{multirow}                                         %%
%%    \usepackage{hhline}                                           %%
%%    \usepackage{ifthen}                                           %%
%%  optionally (for landscape tables embedded in another document): %%
%%    \usepackage{lscape}                                           %%
%%                                                                  %%
%%%%%%%%%%%%%%%%%%%%%%%%%%%%%%%%%%%%%%%%%%%%%%%%%%%%%%%%%%%%%%%%%%%%%%



%%  This section checks if we are begin input into another file or  %%
%%  the file will be compiled alone. First use a macro taken from   %%
%%  the TeXbook ex 7.7 (suggestion of Han-Wen Nienhuys).            %%
\def\ifundefined#1{\expandafter\ifx\csname#1\endcsname\relax}


%%  Check for the \def token for inputed files. If it is not        %%
%%  defined, the file will be processed as a standalone and the     %%
%%  preamble will be used.                                          %%
\ifundefined{inputGnumericTable}

%%  We must be able to close or not the document at the end.        %%
	\def\gnumericTableEnd{\end{document}}


%%%%%%%%%%%%%%%%%%%%%%%%%%%%%%%%%%%%%%%%%%%%%%%%%%%%%%%%%%%%%%%%%%%%%%
%%                                                                  %%
%%  This is the PREAMBLE. Change these values to get the right      %%
%%  paper size and other niceties.                                  %%
%%                                                                  %%
%%%%%%%%%%%%%%%%%%%%%%%%%%%%%%%%%%%%%%%%%%%%%%%%%%%%%%%%%%%%%%%%%%%%%%

	\documentclass[12pt%
			  %,landscape%
                    ]{report}
       \usepackage[latin1]{inputenc}
       \usepackage{fullpage}
       \usepackage{color}
       \usepackage{array}
       \usepackage{longtable}
       \usepackage{calc}
       \usepackage{multirow}
       \usepackage{hhline}
       \usepackage{ifthen}



%%  End of the preamble for the standalone. The next section is for %%
%%  documents which are included into other LaTeX2e files.          %%
\else

%%  We are not a stand alone document. For a regular table, we will %%
%%  have no preamble and only define the closing to mean nothing.   %%
    \def\gnumericTableEnd{}

%%  If we want landscape mode in an embedded document, comment out  %%
%%  the line above and uncomment the two below. The table will      %%
%%  begin on a new page and run in landscape mode.                  %%
%       \def\gnumericTableEnd{\end{landscape}}
%       \begin{landscape}


%%  End of the else clause for this file being \input.              %%
\fi

%%%%%%%%%%%%%%%%%%%%%%%%%%%%%%%%%%%%%%%%%%%%%%%%%%%%%%%%%%%%%%%%%%%%%%
%%                                                                  %%
%%  The rest is the gnumeric table, except for the closing          %%
%%  statement. Changes below will alter the table's appearance.     %%
%%                                                                  %%
%%%%%%%%%%%%%%%%%%%%%%%%%%%%%%%%%%%%%%%%%%%%%%%%%%%%%%%%%%%%%%%%%%%%%%

\providecommand{\gnumericmathit}[1]{#1} 
%%  Uncomment the next line if you would like your numbers to be in %%
%%  italics if they are italizised in the gnumeric table.           %%
%\renewcommand{\gnumericmathit}[1]{\mathit{#1}}
\providecommand{\gnumericPB}[1]%
{\let\gnumericTemp=\\#1\let\\=\gnumericTemp\hspace{0pt}}
 \ifundefined{gnumericTableWidthDefined}
        \newlength{\gnumericTableWidth}
        \newlength{\gnumericTableWidthComplete}
        \newlength{\gnumericMultiRowLength}
        \global\def\gnumericTableWidthDefined{}
 \fi
%% The following setting protects this code from babel shorthands.  %%
 \ifthenelse{\isundefined{\languageshorthands}}{}{\languageshorthands{english}}
%%  The default table format retains the relative column widths of  %%
%%  gnumeric. They can easily be changed to c, r or l. In that case %%
%%  you may want to comment out the next line and uncomment the one %%
%%  thereafter                                                      %%
\providecommand\gnumbox{\makebox[0pt]}
%%\providecommand\gnumbox[1][]{\makebox}

%% to adjust positions in multirow situations                       %%
\setlength{\bigstrutjot}{\jot}
\setlength{\extrarowheight}{\doublerulesep}

%%  The \setlongtables command keeps column widths the same across  %%
%%  pages. Simply comment out next line for varying column widths.  %%
\setlongtables

\setlength\gnumericTableWidth{%
	53pt+%
	163pt+%
0pt}
\def\gumericNumCols{2}
\setlength\gnumericTableWidthComplete{\gnumericTableWidth+%
         \tabcolsep*\gumericNumCols*2+\arrayrulewidth*\gumericNumCols}
\ifthenelse{\lengthtest{\gnumericTableWidthComplete > \linewidth}}%
         {\def\gnumericScale{\ratio{\linewidth-%
                        \tabcolsep*\gumericNumCols*2-%
                        \arrayrulewidth*\gumericNumCols}%
{\gnumericTableWidth}}}%
{\def\gnumericScale{1}}

%%%%%%%%%%%%%%%%%%%%%%%%%%%%%%%%%%%%%%%%%%%%%%%%%%%%%%%%%%%%%%%%%%%%%%
%%                                                                  %%
%% The following are the widths of the various columns. We are      %%
%% defining them here because then they are easier to change.       %%
%% Depending on the cell formats we may use them more than once.    %%
%%                                                                  %%
%%%%%%%%%%%%%%%%%%%%%%%%%%%%%%%%%%%%%%%%%%%%%%%%%%%%%%%%%%%%%%%%%%%%%%

\ifthenelse{\isundefined{\gnumericColA}}{\newlength{\gnumericColA}}{}\settowidth{\gnumericColA}{\begin{tabular}{@{}p{56pt*\gnumericScale}@{}}x\end{tabular}}
\ifthenelse{\isundefined{\gnumericColB}}{\newlength{\gnumericColB}}{}\settowidth{\gnumericColB}{\begin{tabular}{@{}p{163pt*\gnumericScale}@{}}x\end{tabular}}

\begin{tabular}[c]{%
	b{\gnumericColA}%
	b{\gnumericColB}%
	}

%%%%%%%%%%%%%%%%%%%%%%%%%%%%%%%%%%%%%%%%%%%%%%%%%%%%%%%%%%%%%%%%%%%%%%
%%  The longtable options. (Caption, headers... see Goosens, p.124) %%
%	\caption{The Table Caption.}             \\	%
% \hline	% Across the top of the table.
%%  The rest of these options are table rows which are placed on    %%
%%  the first, last or every page. Use \multicolumn if you want.    %%

%%  Header for the first page.                                      %%
%	\multicolumn{2}{c}{The First Header} \\ \hline 
%	\multicolumn{1}{c}{colTag}	%Column 1
%	&\multicolumn{1}{c}{colTag}	\\ \hline %Last column
%	\endfirsthead

%%  The running header definition.                                  %%
%	\hline
%	\multicolumn{2}{l}{\ldots\small\slshape continued} \\ \hline
%	\multicolumn{1}{c}{colTag}	%Column 1
%	&\multicolumn{1}{c}{colTag}	\\ \hline %Last column
%	\endhead

%%  The running footer definition.                                  %%
%	\hline
%	\multicolumn{2}{r}{\small\slshape continued\ldots} \\
%	\endfoot

%%  The ending footer definition.                                   %%
%	\multicolumn{2}{c}{That's all folks} \\ \hline 
%	\endlastfoot
%%%%%%%%%%%%%%%%%%%%%%%%%%%%%%%%%%%%%%%%%%%%%%%%%%%%%%%%%%%%%%%%%%%%%%

\hhline{|-|-}
	 \multicolumn{1}{|p{\gnumericColA}|}%
	{\gnumericPB{\centering}\gnumbox{\textbf{Component}}}
	&\multicolumn{1}{p{\gnumericColB}|}%
	{\gnumericPB{\centering}\gnumbox{\textbf{Description}}}
\\
\hhline{|--|}
	 \multicolumn{1}{|p{\gnumericColA}|}%
	{\gnumericPB{\raggedright}\gnumbox[l]{$R_{id}$}}
	&\multicolumn{1}{p{\gnumericColB}|}%
	{\gnumericPB{\raggedright}\gnumbox[l]{Input Resistance of Op Amp}}
\\
\hhline{|--|}
	 \multicolumn{1}{|p{\gnumericColA}|}%
	{\gnumericPB{\raggedright}\gnumbox[l]{$R_{out}$}}
	&\multicolumn{1}{p{\gnumericColB}|}%
	{\gnumericPB{\raggedright}\gnumbox[l]{Output Resistance of Op Amp}}
\\
\hhline{|--|}
	 \multicolumn{1}{|p{\gnumericColA}|}%
	{\gnumericPB{\raggedright}\gnumbox[l]{$I_s$}}
	&\multicolumn{1}{p{\gnumericColB}|}%
	{\gnumericPB{\raggedright}\gnumbox[l]{Input Current}}
\\
\hhline{|--|}
	 \multicolumn{1}{|p{\gnumericColA}|}%
	{\gnumericPB{\raggedright}\gnumbox[l]{$I_o$}}
	&\multicolumn{1}{p{\gnumericColB}|}%
	{\gnumericPB{\raggedright}\gnumbox[l]{Output Current}}
	
\\
\hhline{|--|}
	 \multicolumn{1}{|p{\gnumericColA}|}%
	{\gnumericPB{\raggedright}\gnumbox[l]{$R_M, R_F$}}
	&\multicolumn{1}{p{\gnumericColB}|}%
	{\gnumericPB{\raggedright}\gnumbox[l]{Feedback Resistances}}
	
\\
\hhline{|--|}
	 \multicolumn{1}{|p{\gnumericColA}|}%
	{\gnumericPB{\raggedright}\gnumbox[l]{$R_L$}}
	&\multicolumn{1}{p{\gnumericColB}|}%
	{\gnumericPB{\raggedright}\gnumbox[l]{Load Resistance}}

\\
\hhline{|-|-|}
\end{tabular}

\ifthenelse{\isundefined{\languageshorthands}}{}{\languageshorthands{\languagename}}
\gnumericTableEnd
    \caption{}
    \label{table:ep18btech11016_comps}
\end{table}

\begin{figure}[!ht]
	\begin{center}
		\resizebox{\columnwidth}{!}{\begin{circuitikz}
\ctikzset{bipoles/length=0.5cm}

\draw 
(0,0) to[R=\tiny$R_F$, i=\tiny$I_f$] (2,0)
(0,0) --(0,-1) --(0.3,-1) to[R=\tiny$R_{id}$, i=\tiny$I_i$] (0.3,-2) node[ground]{}
(0,-1) --(-0.3,-1)
(-0.3,-2) node[ground]{} (-0.3,-2) to[I>=\tiny$I_s$] (-0.3,-1)
(2,0) --(2, -1) to[R, l_=\tiny$R_L$] (1.5,-1) to[R, l_=\tiny$r_o$, i_=\tiny$I_o$] (0.9,-1)
(0.9,-1) to[cI=\tiny$\mu{I_i}$] (0.9,-2) node[ground]{}
(2,-1) --(2.2,-1) to[R, l=\tiny$R_M$, i<=\tiny$I_1$] (2.2,-2) node[ground]{}
node at(2.3,-0.9){\tiny$V_o$}
node at(0.4,-1){\tiny$V$}

;\end{circuitikz}}
	\end{center}
\caption{}
\label{fig:ep18btech11016_equi}
\end{figure}

%--------------------------------------------------------%

\item Given G (open-loop gain), H (feedback gain) and T (closed-loop gain) as
\begin{align}
    G = \cfrac{I_o}{I_i}\\
    H = \cfrac{I_f}{I_o}\\
    T = \cfrac{I_o}{I_f}
\end{align}
Find G and H as a function of the resistances.
\\
\solution Refer fig. \ref{fig:ep18btech11016_equi},\\
We get,
\begin{align}
    I_o = \mu I_i
\end{align}
\begin{align}
    \implies G = \cfrac{I_o}{I_i} = \mu
    \label{eq:open_loop_gain}
\end{align}

And,
\begin{align}
    I_i = I_s - I_f
\end{align}

Using eq. \ref{eq:open_loop_gain}, we get,
\begin{align}
    I_o = \mu \brak {I_s - I_f}
    \label{eq:out_cur}
\end{align}

Assuming virtual ground at $V_1$, we get,
\begin{align}
    V_o = - I_f R_F
\end{align}

Therefore,
\begin{align}
    I_1 = -\cfrac{V_o}{R_M}\\
    \implies I_1 = I_f \cfrac{R_F}{R_M}
\end{align}

The output current is also expressed as,
\begin{align}
    I_o = I_f + I_1\\
    \implies \cfrac{I_o}{1 + \cfrac{R_F}{R_M}} = I_f
\end{align}

Now substituting $I_f$ in  eq. \ref{eq:out_cur} we get,
\begin{align}
    I_o = \mu \brak {I_s - \cfrac{I_o}{1 + \cfrac{R_F}{R_M}}}\\
\end{align}
\begin{align}
    \implies T = \cfrac{I_o}{I_s} = \cfrac{\mu}{1 + \cfrac{\mu}{1 + \cfrac{R_F}{R_M}}}
    \label{eq:closed_loop_gain}
\end{align}

Therefore,
\begin{align}
    \implies H = \cfrac{I_f}{I_o} = \cfrac{ 1 }{1 + \cfrac{R_F}{R_M}}
    \label{eq:feedback_gain}
\end{align}

%--------------------------------------------------------%

\item What will be closed-loop gain(T) if $\mu \to \infty$
\\
\solution From eq. \ref{eq:closed_loop_gain} we get,

\begin{align}
    T = \cfrac{\mu}{1 + \cfrac{\mu}{1 + \cfrac{R_F}{R_M}}}\\
    T = \cfrac{1}{\cfrac{1}{\mu} + \cfrac{1}{1 + \cfrac{R_F}{R_M}}}
\end{align}

Applying the limit, we get,
\begin{align}
    \implies T = 1 + \cfrac{R_F}{R_M}
\end{align}

%--------------------------------------------------------%

\item Refer table \ref{table:ep18btech11016_values} and find G, H and T

\begin{table}[!ht]
    \centering
    %%%%%%%%%%%%%%%%%%%%%%%%%%%%%%%%%%%%%%%%%%%%%%%%%%%%%%%%%%%%%%%%%%%%%%
%%                                                                  %%
%%  This is the header of a LaTeX2e file exported from Gnumeric.    %%
%%                                                                  %%
%%  This file can be compiled as it stands or included in another   %%
%%  LaTeX document. The table is based on the longtable package so  %%
%%  the longtable options (headers, footers...) can be set in the   %%
%%  preamble section below (see PRAMBLE).                           %%
%%                                                                  %%
%%  To include the file in another, the following two lines must be %%
%%  in the including file:                                          %%
%%        \def\inputGnumericTable{}                                 %%
%%  at the beginning of the file and:                               %%
%%        \input{name-of-this-file.tex}                             %%
%%  where the table is to be placed. Note also that the including   %%
%%  file must use the following packages for the table to be        %%
%%  rendered correctly:                                             %%
%%    \usepackage[latin1]{inputenc}                                 %%
%%    \usepackage{color}                                            %%
%%    \usepackage{array}                                            %%
%%    \usepackage{longtable}                                        %%
%%    \usepackage{calc}                                             %%
%%    \usepackage{multirow}                                         %%
%%    \usepackage{hhline}                                           %%
%%    \usepackage{ifthen}                                           %%
%%  optionally (for landscape tables embedded in another document): %%
%%    \usepackage{lscape}                                           %%
%%                                                                  %%
%%%%%%%%%%%%%%%%%%%%%%%%%%%%%%%%%%%%%%%%%%%%%%%%%%%%%%%%%%%%%%%%%%%%%%



%%  This section checks if we are begin input into another file or  %%
%%  the file will be compiled alone. First use a macro taken from   %%
%%  the TeXbook ex 7.7 (suggestion of Han-Wen Nienhuys).            %%
\def\ifundefined#1{\expandafter\ifx\csname#1\endcsname\relax}


%%  Check for the \def token for inputed files. If it is not        %%
%%  defined, the file will be processed as a standalone and the     %%
%%  preamble will be used.                                          %%
\ifundefined{inputGnumericTable}

%%  We must be able to close or not the document at the end.        %%
	\def\gnumericTableEnd{\end{document}}


%%%%%%%%%%%%%%%%%%%%%%%%%%%%%%%%%%%%%%%%%%%%%%%%%%%%%%%%%%%%%%%%%%%%%%
%%                                                                  %%
%%  This is the PREAMBLE. Change these values to get the right      %%
%%  paper size and other niceties.                                  %%
%%                                                                  %%
%%%%%%%%%%%%%%%%%%%%%%%%%%%%%%%%%%%%%%%%%%%%%%%%%%%%%%%%%%%%%%%%%%%%%%

	\documentclass[12pt%
			  %,landscape%
                    ]{report}
       \usepackage[latin1]{inputenc}
       \usepackage{fullpage}
       \usepackage{color}
       \usepackage{array}
       \usepackage{longtable}
       \usepackage{calc}
       \usepackage{multirow}
       \usepackage{hhline}
       \usepackage{ifthen}



%%  End of the preamble for the standalone. The next section is for %%
%%  documents which are included into other LaTeX2e files.          %%
\else

%%  We are not a stand alone document. For a regular table, we will %%
%%  have no preamble and only define the closing to mean nothing.   %%
    \def\gnumericTableEnd{}

%%  If we want landscape mode in an embedded document, comment out  %%
%%  the line above and uncomment the two below. The table will      %%
%%  begin on a new page and run in landscape mode.                  %%
%       \def\gnumericTableEnd{\end{landscape}}
%       \begin{landscape}


%%  End of the else clause for this file being \input.              %%
\fi

%%%%%%%%%%%%%%%%%%%%%%%%%%%%%%%%%%%%%%%%%%%%%%%%%%%%%%%%%%%%%%%%%%%%%%
%%                                                                  %%
%%  The rest is the gnumeric table, except for the closing          %%
%%  statement. Changes below will alter the table's appearance.     %%
%%                                                                  %%
%%%%%%%%%%%%%%%%%%%%%%%%%%%%%%%%%%%%%%%%%%%%%%%%%%%%%%%%%%%%%%%%%%%%%%

\providecommand{\gnumericmathit}[1]{#1} 
%%  Uncomment the next line if you would like your numbers to be in %%
%%  italics if they are italizised in the gnumeric table.           %%
%\renewcommand{\gnumericmathit}[1]{\mathit{#1}}
\providecommand{\gnumericPB}[1]%
{\let\gnumericTemp=\\#1\let\\=\gnumericTemp\hspace{0pt}}
 \ifundefined{gnumericTableWidthDefined}
        \newlength{\gnumericTableWidth}
        \newlength{\gnumericTableWidthComplete}
        \newlength{\gnumericMultiRowLength}
        \global\def\gnumericTableWidthDefined{}
 \fi
%% The following setting protects this code from babel shorthands.  %%
 \ifthenelse{\isundefined{\languageshorthands}}{}{\languageshorthands{english}}
%%  The default table format retains the relative column widths of  %%
%%  gnumeric. They can easily be changed to c, r or l. In that case %%
%%  you may want to comment out the next line and uncomment the one %%
%%  thereafter                                                      %%
\providecommand\gnumbox{\makebox[0pt]}
%%\providecommand\gnumbox[1][]{\makebox}

%% to adjust positions in multirow situations                       %%
\setlength{\bigstrutjot}{\jot}
\setlength{\extrarowheight}{\doublerulesep}

%%  The \setlongtables command keeps column widths the same across  %%
%%  pages. Simply comment out next line for varying column widths.  %%
\setlongtables

\setlength\gnumericTableWidth{%
	53pt+%
	163pt+%
0pt}
\def\gumericNumCols{2}
\setlength\gnumericTableWidthComplete{\gnumericTableWidth+%
         \tabcolsep*\gumericNumCols*2+\arrayrulewidth*\gumericNumCols}
\ifthenelse{\lengthtest{\gnumericTableWidthComplete > \linewidth}}%
         {\def\gnumericScale{\ratio{\linewidth-%
                        \tabcolsep*\gumericNumCols*2-%
                        \arrayrulewidth*\gumericNumCols}%
{\gnumericTableWidth}}}%
{\def\gnumericScale{1}}

%%%%%%%%%%%%%%%%%%%%%%%%%%%%%%%%%%%%%%%%%%%%%%%%%%%%%%%%%%%%%%%%%%%%%%
%%                                                                  %%
%% The following are the widths of the various columns. We are      %%
%% defining them here because then they are easier to change.       %%
%% Depending on the cell formats we may use them more than once.    %%
%%                                                                  %%
%%%%%%%%%%%%%%%%%%%%%%%%%%%%%%%%%%%%%%%%%%%%%%%%%%%%%%%%%%%%%%%%%%%%%%

\ifthenelse{\isundefined{\gnumericColA}}{\newlength{\gnumericColA}}{}\settowidth{\gnumericColA}{\begin{tabular}{@{}p{60pt*\gnumericScale}@{}}x\end{tabular}}
\ifthenelse{\isundefined{\gnumericColB}}{\newlength{\gnumericColB}}{}\settowidth{\gnumericColB}{\begin{tabular}{@{}p{40pt*\gnumericScale}@{}}x\end{tabular}}

\begin{tabular}[c]{%
	b{\gnumericColA}%
	b{\gnumericColB}%
	}

%%%%%%%%%%%%%%%%%%%%%%%%%%%%%%%%%%%%%%%%%%%%%%%%%%%%%%%%%%%%%%%%%%%%%%
%%  The longtable options. (Caption, headers... see Goosens, p.124) %%
%	\caption{The Table Caption.}             \\	%
% \hline	% Across the top of the table.
%%  The rest of these options are table rows which are placed on    %%
%%  the first, last or every page. Use \multicolumn if you want.    %%

%%  Header for the first page.                                      %%
%	\multicolumn{2}{c}{The First Header} \\ \hline 
%	\multicolumn{1}{c}{colTag}	%Column 1
%	&\multicolumn{1}{c}{colTag}	\\ \hline %Last column
%	\endfirsthead

%%  The running header definition.                                  %%
%	\hline
%	\multicolumn{2}{l}{\ldots\small\slshape continued} \\ \hline
%	\multicolumn{1}{c}{colTag}	%Column 1
%	&\multicolumn{1}{c}{colTag}	\\ \hline %Last column
%	\endhead

%%  The running footer definition.                                  %%
%	\hline
%	\multicolumn{2}{r}{\small\slshape continued\ldots} \\
%	\endfoot

%%  The ending footer definition.                                   %%
%	\multicolumn{2}{c}{That's all folks} \\ \hline 
%	\endlastfoot
%%%%%%%%%%%%%%%%%%%%%%%%%%%%%%%%%%%%%%%%%%%%%%%%%%%%%%%%%%%%%%%%%%%%%%

\hhline{|-|-}
	 \multicolumn{1}{|p{\gnumericColA}|}%
	{\gnumericPB{\centering}\gnumbox{\textbf{Component}}}
	&\multicolumn{1}{p{\gnumericColB}|}%
	{\gnumericPB{\centering}\gnumbox{\textbf{Value}}}
\\
\hhline{|--|}
	 \multicolumn{1}{|p{\gnumericColA}|}%
	{\gnumericPB{\raggedright}\gnumbox[l]{$\mu$}}
	&\multicolumn{1}{p{\gnumericColB}|}%
	{\gnumericPB{\raggedright}\gnumbox[l]{$10^4$}}
\\
\hhline{|--|}
	 \multicolumn{1}{|p{\gnumericColA}|}%
	{\gnumericPB{\raggedright}\gnumbox[l]{$R_{id}$}}
	&\multicolumn{1}{p{\gnumericColB}|}%
	{\gnumericPB{\raggedright}\gnumbox[l]{1 $M\Omega$}}
\\
\hhline{|--|}
	 \multicolumn{1}{|p{\gnumericColA}|}%
	{\gnumericPB{\raggedright}\gnumbox[l]{$r_o$}}
	&\multicolumn{1}{p{\gnumericColB}|}%
	{\gnumericPB{\raggedright}\gnumbox[l]{100 $\Omega$}}
\\
\hhline{|--|}
	 \multicolumn{1}{|p{\gnumericColA}|}%
	{\gnumericPB{\raggedright}\gnumbox[l]{$R_L$}}
	&\multicolumn{1}{p{\gnumericColB}|}%
	{\gnumericPB{\raggedright}\gnumbox[l]{10 k$\Omega$}}
	
\\
\hhline{|--|}
	 \multicolumn{1}{|p{\gnumericColA}|}%
	{\gnumericPB{\raggedright}\gnumbox[l]{$R_M$}}
	&\multicolumn{1}{p{\gnumericColB}|}%
	{\gnumericPB{\raggedright}\gnumbox[l]{100 $\Omega$}}
	
\\
\hhline{|--|}
	 \multicolumn{1}{|p{\gnumericColA}|}%
	{\gnumericPB{\raggedright}\gnumbox[l]{$R_F$}}
	&\multicolumn{1}{p{\gnumericColB}|}%
	{\gnumericPB{\raggedright}\gnumbox[l]{10 k$\Omega$}}

\\
\hhline{|-|-|}
\end{tabular}

\ifthenelse{\isundefined{\languageshorthands}}{}{\languageshorthands{\languagename}}
\gnumericTableEnd
    \caption{}
    \label{table:ep18btech11016_values}
\end{table}

\solution Using eqs. \ref{eq:open_loop_gain}, \ref{eq:feedback_gain} and \ref{eq:closed_loop_gain}

We get,
\begin{align}
    G = \mu = 10^4\\
    H = \cfrac{ 1 }{1 + \cfrac{R_F}{R_M}} = 9.9 \times 10^{-3}\\
    T = \cfrac{\mu}{1 + \cfrac{\mu}{1 + \cfrac{R_F}{R_M}}} = 100
\end{align}

%--------------------------------------------------------%

\item Tabulate your results.
\\
\solution Refer table \ref{table:ep18btech11016_final},

\begin{table}[!ht]
    \centering
    %%%%%%%%%%%%%%%%%%%%%%%%%%%%%%%%%%%%%%%%%%%%%%%%%%%%%%%%%%%%%%%%%%%%%%
%%                                                                  %%
%%  This is the header of a LaTeX2e file exported from Gnumeric.    %%
%%                                                                  %%
%%  This file can be compiled as it stands or included in another   %%
%%  LaTeX document. The table is based on the longtable package so  %%
%%  the longtable options (headers, footers...) can be set in the   %%
%%  preamble section below (see PRAMBLE).                           %%
%%                                                                  %%
%%  To include the file in another, the following two lines must be %%
%%  in the including file:                                          %%
%%        \def\inputGnumericTable{}                                 %%
%%  at the beginning of the file and:                               %%
%%        \input{name-of-this-file.tex}                             %%
%%  where the table is to be placed. Note also that the including   %%
%%  file must use the following packages for the table to be        %%
%%  rendered correctly:                                             %%
%%    \usepackage[latin1]{inputenc}                                 %%
%%    \usepackage{color}                                            %%
%%    \usepackage{array}                                            %%
%%    \usepackage{longtable}                                        %%
%%    \usepackage{calc}                                             %%
%%    \usepackage{multirow}                                         %%
%%    \usepackage{hhline}                                           %%
%%    \usepackage{ifthen}                                           %%
%%  optionally (for landscape tables embedded in another document): %%
%%    \usepackage{lscape}                                           %%
%%                                                                  %%
%%%%%%%%%%%%%%%%%%%%%%%%%%%%%%%%%%%%%%%%%%%%%%%%%%%%%%%%%%%%%%%%%%%%%%



%%  This section checks if we are begin input into another file or  %%
%%  the file will be compiled alone. First use a macro taken from   %%
%%  the TeXbook ex 7.7 (suggestion of Han-Wen Nienhuys).            %%
\def\ifundefined#1{\expandafter\ifx\csname#1\endcsname\relax}


%%  Check for the \def token for inputed files. If it is not        %%
%%  defined, the file will be processed as a standalone and the     %%
%%  preamble will be used.                                          %%
\ifundefined{inputGnumericTable}

%%  We must be able to close or not the document at the end.        %%
	\def\gnumericTableEnd{\end{document}}


%%%%%%%%%%%%%%%%%%%%%%%%%%%%%%%%%%%%%%%%%%%%%%%%%%%%%%%%%%%%%%%%%%%%%%
%%                                                                  %%
%%  This is the PREAMBLE. Change these values to get the right      %%
%%  paper size and other niceties.                                  %%
%%                                                                  %%
%%%%%%%%%%%%%%%%%%%%%%%%%%%%%%%%%%%%%%%%%%%%%%%%%%%%%%%%%%%%%%%%%%%%%%

	\documentclass[12pt%
			  %,landscape%
                    ]{report}
       \usepackage[latin1]{inputenc}
       \usepackage{fullpage}
       \usepackage{color}
       \usepackage{array}
       \usepackage{longtable}
       \usepackage{calc}
       \usepackage{multirow}
       \usepackage{hhline}
       \usepackage{ifthen}



%%  End of the preamble for the standalone. The next section is for %%
%%  documents which are included into other LaTeX2e files.          %%
\else

%%  We are not a stand alone document. For a regular table, we will %%
%%  have no preamble and only define the closing to mean nothing.   %%
    \def\gnumericTableEnd{}

%%  If we want landscape mode in an embedded document, comment out  %%
%%  the line above and uncomment the two below. The table will      %%
%%  begin on a new page and run in landscape mode.                  %%
%       \def\gnumericTableEnd{\end{landscape}}
%       \begin{landscape}


%%  End of the else clause for this file being \input.              %%
\fi

%%%%%%%%%%%%%%%%%%%%%%%%%%%%%%%%%%%%%%%%%%%%%%%%%%%%%%%%%%%%%%%%%%%%%%
%%                                                                  %%
%%  The rest is the gnumeric table, except for the closing          %%
%%  statement. Changes below will alter the table's appearance.     %%
%%                                                                  %%
%%%%%%%%%%%%%%%%%%%%%%%%%%%%%%%%%%%%%%%%%%%%%%%%%%%%%%%%%%%%%%%%%%%%%%

\providecommand{\gnumericmathit}[1]{#1} 
%%  Uncomment the next line if you would like your numbers to be in %%
%%  italics if they are italizised in the gnumeric table.           %%
%\renewcommand{\gnumericmathit}[1]{\mathit{#1}}
\providecommand{\gnumericPB}[1]%
{\let\gnumericTemp=\\#1\let\\=\gnumericTemp\hspace{0pt}}
 \ifundefined{gnumericTableWidthDefined}
        \newlength{\gnumericTableWidth}
        \newlength{\gnumericTableWidthComplete}
        \newlength{\gnumericMultiRowLength}
        \global\def\gnumericTableWidthDefined{}
 \fi
%% The following setting protects this code from babel shorthands.  %%
 \ifthenelse{\isundefined{\languageshorthands}}{}{\languageshorthands{english}}
%%  The default table format retains the relative column widths of  %%
%%  gnumeric. They can easily be changed to c, r or l. In that case %%
%%  you may want to comment out the next line and uncomment the one %%
%%  thereafter                                                      %%
\providecommand\gnumbox{\makebox[0pt]}
%%\providecommand\gnumbox[1][]{\makebox}

%% to adjust positions in multirow situations                       %%
\setlength{\bigstrutjot}{\jot}
\setlength{\extrarowheight}{\doublerulesep}

%%  The \setlongtables command keeps column widths the same across  %%
%%  pages. Simply comment out next line for varying column widths.  %%
\setlongtables

\setlength\gnumericTableWidth{%
	53pt+%
	163pt+%
0pt}
\def\gumericNumCols{2}
\setlength\gnumericTableWidthComplete{\gnumericTableWidth+%
         \tabcolsep*\gumericNumCols*2+\arrayrulewidth*\gumericNumCols}
\ifthenelse{\lengthtest{\gnumericTableWidthComplete > \linewidth}}%
         {\def\gnumericScale{\ratio{\linewidth-%
                        \tabcolsep*\gumericNumCols*2-%
                        \arrayrulewidth*\gumericNumCols}%
{\gnumericTableWidth}}}%
{\def\gnumericScale{1}}

%%%%%%%%%%%%%%%%%%%%%%%%%%%%%%%%%%%%%%%%%%%%%%%%%%%%%%%%%%%%%%%%%%%%%%
%%                                                                  %%
%% The following are the widths of the various columns. We are      %%
%% defining them here because then they are easier to change.       %%
%% Depending on the cell formats we may use them more than once.    %%
%%                                                                  %%
%%%%%%%%%%%%%%%%%%%%%%%%%%%%%%%%%%%%%%%%%%%%%%%%%%%%%%%%%%%%%%%%%%%%%%

\ifthenelse{\isundefined{\gnumericColA}}{\newlength{\gnumericColA}}{}\settowidth{\gnumericColA}{\begin{tabular}{@{}p{50pt*\gnumericScale}@{}}x\end{tabular}}
\ifthenelse{\isundefined{\gnumericColB}}{\newlength{\gnumericColB}}{}\settowidth{\gnumericColB}{\begin{tabular}{@{}p{100pt*\gnumericScale}@{}}x\end{tabular}}

\begin{tabular}[c]{%
	b{\gnumericColA}%
	b{\gnumericColB}%
	}

%%%%%%%%%%%%%%%%%%%%%%%%%%%%%%%%%%%%%%%%%%%%%%%%%%%%%%%%%%%%%%%%%%%%%%
%%  The longtable options. (Caption, headers... see Goosens, p.124) %%
%	\caption{The Table Caption.}             \\	%
% \hline	% Across the top of the table.
%%  The rest of these options are table rows which are placed on    %%
%%  the first, last or every page. Use \multicolumn if you want.    %%

%%  Header for the first page.                                      %%
%	\multicolumn{2}{c}{The First Header} \\ \hline 
%	\multicolumn{1}{c}{colTag}	%Column 1
%	&\multicolumn{1}{c}{colTag}	\\ \hline %Last column
%	\endfirsthead

%%  The running header definition.                                  %%
%	\hline
%	\multicolumn{2}{l}{\ldots\small\slshape continued} \\ \hline
%	\multicolumn{1}{c}{colTag}	%Column 1
%	&\multicolumn{1}{c}{colTag}	\\ \hline %Last column
%	\endhead

%%  The running footer definition.                                  %%
%	\hline
%	\multicolumn{2}{r}{\small\slshape continued\ldots} \\
%	\endfoot

%%  The ending footer definition.                                   %%
%	\multicolumn{2}{c}{That's all folks} \\ \hline 
%	\endlastfoot
%%%%%%%%%%%%%%%%%%%%%%%%%%%%%%%%%%%%%%%%%%%%%%%%%%%%%%%%%%%%%%%%%%%%%%

\hhline{|-|-}
	 \multicolumn{1}{|p{\gnumericColA}|}%
	{\gnumericPB{\centering}\gnumbox{\textbf{Gain}}}
	&\multicolumn{1}{p{\gnumericColB}|}%
	{\gnumericPB{\centering}\gnumbox{\textbf{Value}}}
\\
\hhline{|--|}
	 \multicolumn{1}{|p{\gnumericColA}|}%
	{\gnumericPB{\raggedright}\gnumbox[l]{G}}
	&\multicolumn{1}{p{\gnumericColB}|}%
	{\gnumericPB{\raggedright}\gnumbox[l]{$10^4$}}
\\
\hhline{|--|}
	 \multicolumn{1}{|p{\gnumericColA}|}%
	{\gnumericPB{\raggedright}\gnumbox[l]{H}}
	&\multicolumn{1}{p{\gnumericColB}|}%
	{\gnumericPB{\raggedright}\gnumbox[l]{$9.9 \times 10^{-3}$}}
\\
\hhline{|--|}
	 \multicolumn{1}{|p{\gnumericColA}|}%
	{\gnumericPB{\raggedright}\gnumbox[l]{T}}
	&\multicolumn{1}{p{\gnumericColB}|}%
	{\gnumericPB{\raggedright}\gnumbox[l]{100}}

\\
\hhline{|-|-|}
\end{tabular}

\ifthenelse{\isundefined{\languageshorthands}}{}{\languageshorthands{\languagename}}
\gnumericTableEnd
    \caption{}
    \label{table:ep18btech11016_final}
\end{table}

\end{enumerate}